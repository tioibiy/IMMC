%! Author = ruochongli
%! Date = 2023/3/11

% Preamble
\documentclass[./main.tex]{subfiles}
\linespread{1.25}
% Packages

% Document
\begin{document}
    \section*{A Letter To Decision Makers}
    \textbf{Dear Community Leaders and Business Planners,}

    \vspace{1.5ex}

    We are writing to provide you with the necessary information regarding the optimal placement of facilities in an area of Red Creek.
    After careful analysis, our team has developed a mathematical model to address the problem.
    The model can take into consideration all metrics related to the problem to assess your plans by simulating the
    interaction between facilities, social and natural conditions.
    On a greater scale, this not only considers the balance between economic benefits and nature conservation, but also involves the consideration of short-term profit along with long-term sustainability.

    Our core idea of the model is the grading system.
    By evaluating the suitability of each facility on each part of the land, we can evaluate plans and provide the optimal plan for you.
    Based on the highest score ($100$ pts) of our best plan, you may provide us with your customized plan so we can assess your plan by Grade A ($85-100$ pts), Grade B ($60-85$ pts), and Grade C ($0-60$ pts) based on our model.

    Our model was designed in the following way.
    We divided the area into nearly 2000 grids ($37\times 54$) and set various factors into each grid including topography, climate, environmental impact, economic benefits, social factors, and such.
    Moreover, our model's algorithm is highly customizable to meet your unique requirements, as we can adjust the
    weighting of specific metrics during calculations, such as slope, humidity, illumination, temperature, visitors
    flow rate, and so on.

    We applied the model to the land and the result is highly correct.
    We artificially decided on some parts of the area for four facilities' possibly suitable locations and used the
    model to calculate the suitability of the plan.
    By doing so, we can identify the best plan by comparing the suitability scores of the artificial plans with and without location swaps.
    According to our results, the former is always higher than the latter, which is a testimony to our man-made plan.
    This can also be applied to your plans as well.

    To verify the accuracy and effectiveness of the model, we tested it on a property in Malaysia by comparing the model's results with the actual choice made by real decision-makers.
    The results showed that the model is highly accurate and effective in determining the optimal location for a sustainable resort.

    Our model has several strengths that make it an ideal solution for determining the optimal arrangements of facilities.
    Firstly, it is highly customizable and can be modified according to specific metrics.
    Secondly, it can involve every metric you can come up with, making it a highly accurate model.

    In conclusion, our team recommends using our mathematical model to evaluate your favored plans.
    We are confident that our model is accurate and effective in assessing your plans and predicting the optimal location based on various factors.
    A diagram of our calculated optimal construction plan (Figure~\ref{fig:figureNNBB}) is attached to Appendix A .

    We look forward to discussing our solution with you further.


    \vspace{1.5ex}

    \textbf{Sincerely,}

    \raggedright
    \textbf{Team IMMC23250946}


\end{document}