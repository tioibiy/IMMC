%! Author = ruochongli
%! Date = 2023/3/10

% Preamble
\documentclass[./main.tex]{subfiles}

% Packages
\linespread{1.5}
% Document
\begin{document}
    \begin{center}
        \large
        \textbf{Summary}
    \end{center}
        Decision makers faced the task of deciding on the best construction plan for a 3-square-kilometer area in Syracuse, NY, USA. The team was required to decide on a series of metrics, collect data and quantitatively evaluate construction plans.
        In response to the task, the team first decided on geological and human metrics according to data.
        After this, the team used differential equations to quantify the mutual influence between facilities' activity, geographical conditions and human factors.
        At last, the team divided the area into a host of grids, and evaluated the suitability of each grid for different facilities in order to give normalized points for different construction plans.

        \vspace{1.25em}

        When it comes to the metrics, the team delved into data and brainstormed a series of metrics of geology and human.
        Geological ones were slope, vegetation coverage, humidity, precipitation, illumination, temperature and so on, while human ones were revenue, maintenance cost, prime cost, employment, visitors flow rate and such.
        Weights are assigned to each metric, and they are highly changeable according to your demands.

        \vspace{1.25em}

        In order to quantify the mutual influence between facilities and the environment, the team took differential equations into consideration.
        Since the changing rate of metrics was obviously dependent on the activity of facilities, their influence and time, the team could use differential equations to well describe and quantify the relations.
        By using differential equations, the team can give precise future predictions and better describe the
        relationship between time and amounts.
        Based on the differential equations, we have weighted integral equations that can quantify the profits.

        \vspace{1.25em}

        Via the model, the team found the best facility construction plan (the highest suitability) for this place and normalized the evaluation results so that the team can better quantitatively assess each grid in the form of scores (out of $100$ pts) of a provided construction plan.
        To apply the model, the team artificially determined two pairs of plans, where each pair included a suitable and an unsuitable arrangements.
        By substituting these two pairs into the model, the team got the result that the suitable plan scored higher
    than the unsuitable one among the two pairs.

        \vspace{1.25em}

        Our model included good scalability, alongside its well refactoring and decoupling capabilities.
        After sensitivity and scalability analysis, the team is confident that this model can respond to different construction needs in different regions, while being ready for adjustments according to policy requirements and regional characteristics with a high degree of personalization.


        \raggedright
        \vspace{2em}
        \keywords Matrix Analysis, Multi-Objective Analysis, Differential Equations and Weighted Integral
    Operation, Environmental Factors



\end{document}