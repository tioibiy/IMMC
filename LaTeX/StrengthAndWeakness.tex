%! Author = ruochongli
%! Date = 2023/3/11

% Preamble
\documentclass[./main.tex]{subfiles}

% Packages

% Document
\begin{document}
    \section{Strength and Weakness}
    \subsection{Strength}
    \begin{enumerate}
        \item Our model divides the property of the entire Red Creek into just short of $2000$ grids, which is able to take
        the difference in topography and environment of grids into consideration of the whole solution.
        This model can eliminate some facilities on steeper slopes to make the solution more realistic.
        Furthermore, the pollution caused by some facilities will distribute in the soil can also be taken into
        consideration.
        Generally, the model can be used to analyze complex and varying terrain.
        \item The basic structure of our overall model allows highly customized needs and a large range of expansion.
        The function of customization is developed because the algorithm $\alpha_{pos, i}\left ( t \right )$ can be
        modified according to a specific metric $i$ such as slope, humidity or temperature.
        Similar modification can be made in $U$ and $V$, which is used to measure the interaction between all the grids to one fixed grid when the range of $i$ is affirmation.
        Furthermore, the model is expandable for the use of nearly every problem whose pivotal difficulty is
         to quantify the impact between considered grids.
        This is because the form of \lq\lq{matrix $\times$ matrix
        }\rq\rq in the model can reflect all the mutual influence between each grid due to the feature of the
        cross multiplication.
        \item The model has the potential to be very accurate because theoretically it can involve every metric that
        the problem requires to consider.
        Also, the sub-function used to quantify the metric is customized, which can also be very
        accurate if there are sufficient data.
    \end{enumerate}
    \subsection{Weakness}
    \begin{enumerate}
        \item The model is simplified.
        It assumes that each grid within a grid is homogeneous, and does not take into account variations within each grid.
        This oversimplification may result in inaccurate results in some scenarios.
        \item The model is difficulty to calibrate.
        The model requires calibration to ensure accurate results, which can be a time-consuming and challenging process.
        \item The model is relatively complex and may require significant technical expertise to develop and operate, making it less accessible to those without specialized training.
    \end{enumerate}



\end{document}