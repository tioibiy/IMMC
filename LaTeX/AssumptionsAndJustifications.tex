%! Author = ruochongli
%! Date = 2023/3/9

% Preamble
\documentclass[./main.tex]{subfiles}
\usepackage{mcmthesis}

% Packages

% Document
\begin{document}
    \section{Assumptions}
    To simplify the problem and make it convenient for us to simulate real-life conditions, we make the following
    basic assumptions, each of which is properly justified. 
    \begin{itemize}
        \item \textbf{Pollutants all follow the convective diffusion equation. }
        The convective diffusion equation is stable and the most widely used equation for particles.
        Considering the pollutants are mostly chemicals, the convective diffusion equation is suitable.
        \item \textbf{All media are isotropic. }
        The main media in the property are soil and air,
        which can be seen as equally mixed - thus they are isotropic media.
        With isotropic media, partial differential equations require less time complexity.
        \item \textbf{If no facilities are built, the population stays stable. }
        According to Wikipedia, the population growth rate of Syracuse is close to $0\%$ from 2000 to 2020.
        With an educated guess, the population will not be able to change the overall solution. 
        \item \textbf{Short-term benefits are those earned in one year. }
        If \lq\lq{short-term}\rq\rq is defined to be less than one year, special facilities such as cross-country skiing
        facility may not be profitable as it opens just for 3 months.
        \item \textbf{Long-term benefits are those benefits earned in ten years. }
        This allows for sufficient time to recover from market fluctuations and benefit from compound interests,
        aligning with some common financial goals such as retirement planning, college savings, or a house purchase.
        This also reduces the impact of taxes and fees on investment returns.
        \item \textbf{The increase in tourists generated by the agglomeration effect is negligible. }
        In this specific condition, there are only two types of facilities that depend on tourists: an agritourist
        center and a cross-country skiing facility.
        These two facilities operate in different seasons, thus they will not interfere with each other by
        agglomeration effect.
        \item \textbf{Facilities and environment influence each other by causing effect on grids. }
        We simplify the mutual influence between facilities and environment to the influence on grids.
        In other words, they affect each other via grids as they depend on conditions of their grids.
        \item \textbf{Geological metrics follow normal distribution. }
        The probability of most events in nature follow the normal distribution, thus it is rather appropriate to
        assume that all geological metrics in this case are the same.
    \end{itemize}

    With the appropriate assumptions and justifications above, we can make mathematical model more feasible to
    build while being realistic as much as possible.




\end{document}